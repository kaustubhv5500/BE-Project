\chapter{Introduction}
Satellite systems play a major role in global communications, being used in television, telephone, radio, internet, and military applications. These systems transmit and receive various data types such as voice, video, geo-location, and TCP/IP packets. These systems are complex and need to be computationally efficient due to the availability of limited resources. Thus, communication systems in these satellites must be efficient and have low bit error rates. To better utilize scarce channel resources and to improve temporal and computational complexity, these systems use multiplexing to process multiple signals at the same time. Multiplexing is a way of sending multiple signals or streams of information over a communications link simultaneously in the form of a single, complex signal. While conventional systems that implement Time and Frequency Division Multiplexing still exist, these offer low data rates at high bit error rates. Filter Bank based solutions have been gaining popularity as these can increase data rates while reducing error rates. The high computational power achieved by onboard processing systems allows these solutions to work efficiently. \par
\section{Trans-multiplexers}
Trans-multiplexers are the application of digital filter banks that are used to convert time-division-multiplexed signals (TDM) to frequency-division-multiplexed (FDM) signals. In a Trans-multiplexer, for TDM-FDM conversion, the input signal {x(n)} is a time-division-multiplexed signal consisting of $N$ signals, which are separated by a commutator switch. Each of the $N$ signals is then interpolated and modulated on a different carrier frequency to obtain an FDM signal for transmission. For FDM-TDM conversion, the composite signal is separated by filtering into N components and decimating which results in TDM signals. Conventionally, Trans-multiplexer systems utilize Cosine and Fourier transformation functions for signal reconstruction and filter design. These however do not provide Perfect Reconstruction of the signals at the output. \par
\newpage
\section{The Wavelet Transform}
To solve the problem of cross-talk due to the loss of time-domain information by these transforms, the use of Wavelet Transform has been proposed. The Wavelet Series represents a square-integrable function by an orthonormal series generated by a mother wavelet. In the Discrete Wavelet Transform (DWT), the wavelets are discretely sampled, preserving both the frequency and location information of the signal. Moreover, the DWT algorithm has a time complexity of $O(N)$ instead of $O(N*log(N))$, thus making the system more efficient. The DWT of a signal is calculated by passing it through a series of filters. First the samples are passed through a low pass filter resulting in a convolution of the two. The signal is also decomposed simultaneously using a high-pass filter. The outputs give the detail and approximation coefficients.



\section{Layout of the Report} A brief chapter by chapter overview is presented here.\\
Chapter 2: A literature review of different methods for Trans-multiplexer design and theory about Wavelet Transform based filter Bank design is presented.  \\
Chapter 3: The main objectives of the project will be described in this chapter.\\
Chapter 4: In this chapter, the algorithm for Wavelet Transform based Trans-multiplexer design is discussed and the designed system parameters are presented.  \\
Chapter 5: The methodology and results of the simulation conducted are presented and discussed here. \\
Chapter 6 \& 7: The project is concluded and its future scope is discussed in these sections respectively.
