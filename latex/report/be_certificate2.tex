\newpage
\thispagestyle{empty}
\vspace*{0.2cm}
\vspace{1cm}
\begin{center}
 \large\textbf{Project approval Certificate}
\end{center}
\vspace{2cm}
This is to certify that the Project entitled ``Multiple Channel Perfect Reconstruction Transmultiplexer using Wavelet Transform Techniques for Satellite Communications" by Mr. Kaustubh Venkatesh, Mr. Gokul Nair and Ms. Arseta Singh is approved  for the award of Degree of Bachelor of Technology in Electronics \& Telecommunication Engineering from Sardar Patel Institute of Technology.\\
\\
\\
\\
\\
\\
\\
\\
\\
\textbf {External Examiner} \hspace{2.85in} \textbf{Internal Examiner}\\ 
\\
\\
\\
\\
\textbf {(signature)} \hspace{3.5in} \textbf{(signature)} \\
\\
\vspace{1cm}
\textbf {Name:} \hspace{3.8in} \textbf{Name:}\\ 
\vspace{2cm}
\textbf {Date:} \hspace{3.9in} \textbf{Date:}\\ 
\vspace{2cm}
\begin{center}
\textbf{Seal of the Institute}\\
\end{center}

\newpage
\thispagestyle{empty} 
\begin{center}
 \large\textbf{Vision}
\end{center}
Today, access to information and rapid communication across the globe has changed the fundamentals of interaction in human society. Satellite systems play an integral role in knitting the global communication network together, thus making developments in the field an important goal of engineering. Recently, the Indian Space Research Organisation (ISRO) has made great advancements in satellite engineering and deployment with a fraction of the budget of other space agencies. Thus, Telecommunication Engineers in India must work towards developing indigenous systems to help expedite the research process. The motivation for this project was part of a research proposal contained in ISRO's Response Basket. This project was developed to meet the requirements stated in the same.

\newpage
\thispagestyle{empty} 
\begin{center}
 \large\textbf{Acknowledgement}
\end{center}
We would like to thank our mentor Dr. Sukanya Kulkarni for her continuous support over the last several months. Her enthusiasm towards the successful development of this project has been extremely important and a constant source of motivation for us. Her guidance and insights throughout the course of this project helped us come up with novel ideas and execute them effectively. Her experience in mentoring students brought out the best in us, and helped us overcome the few roadblocks we faced along the way. We would also like to thank Sardar Patel Institute of Technology for providing us with the required resources which allowed us to freely express our ideas with a great deal of conviction.
\newpage
